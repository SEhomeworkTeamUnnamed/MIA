%!TEX program = xelatex
\documentclass[hyperref, UTF8
,bookmarksnumbered=true]{ctexart}
\hypersetup{
            bookmarksnumbered=true,
            bookmarksopen=true,
            colorlinks=false, 
            pdfborder=001,   
				%menucolor=green,%uunknown
				linktocpage=true,%make the link of the content on the number of page
            linkcolor=green,
            anchorcolor=green,
            citecolor=green}
\usepackage{geometry}
\usepackage{tocbibind}
\usepackage{graphicx}
\usepackage{amsmath}
\usepackage{amsfonts}
\usepackage{amssymb}
\usepackage{bm}
\CTEXsetup[name={(,)}]{chapter}
%\CTEXsetup[number={\chinese{section}}]{section}


\newcommand{\beqt}{\begin{equation}}
\newcommand{\eeqt}{\end{equation}}
\newcommand{\beqtnt}{\begin{equation*}}
\newcommand{\eeqtnt}{\end{equation*}}

\geometry{a4paper,centering,scale=0.8}

\title{软件工程课程总结}

\author{基科物理32{  }蒋文韬{  }2013011717}

\begin{document}\large
	\maketitle
      经过了一学期的软件工程课程的学习,我对软件工程的 课程内容有了较为全面的了解,从软件开发与维护的整个过程认识了软件工程相关知识的重要性。如需求分析中的各种方法,逐步细化与模块化的设计方法,面向对象对程序的类架构进行设计,以及对软件进行测试。在之前的程序相关课程中,虽然都或大或小地完成过编程大作业,但从来没有像软件工程课程这样以较为规范的软件开发流程来执行软件需求,设计,编写,测试等步骤。这门课程的大作业也是我第一次与其他三人共同合作,并且承担组长的职务。

      组长的角色让我更加深刻地认识到了软件开发的相关理念在实际开发过程中的重要性。比如设计部分对不同类的功能的描述的细致程度以及对类的接口描述的细致程度,直接决定了负责不同模块的程序员编写出来的程序能否合同起来成功工作。对用户界面的功能的定义与具体的设计,也直接决定开始编写程序后再次进行讨论与商议的时间与精力。由于是第一次学习软件工程的有关概念,技术与方法,并且也是第一次将这些方法实际使用,因此很多地方我们都理解的不是十分到位,也做得不是十分到位,例如设计部分的详细与准确程度就十分差,导致了实际编写代码过程中组员之间经常再对相关内容进行讨论与商议。测试方面我们也因为时间与精力有限的缘故没有使用较为专业的GUI测试程序。

      我们开发使用的语言为Java,图形界面在Java Swing的基础上完成。我们的组员之前均未接触过Java语言,因此这方面也造成了一定困扰。不过通过每周定时开会讨论与任务检查与分配,经常性的微信线上交流与讨论,我们均解决了实际开发过程中遇到的众多问题,最后成功实现了需求中设计好的功能,整体合作还是比较成功的。

      最后,感谢老师与助教一学期的辛勤付出。软件工程这门课让我收获颇丰,更是认识到了软件工程的重要性。
\end{document}